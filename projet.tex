% Options for packages loaded elsewhere
\PassOptionsToPackage{unicode}{hyperref}
\PassOptionsToPackage{hyphens}{url}
%
\documentclass[
]{article}
\usepackage{lmodern}
\usepackage{amssymb,amsmath}
\usepackage{ifxetex,ifluatex}
\ifnum 0\ifxetex 1\fi\ifluatex 1\fi=0 % if pdftex
  \usepackage[T1]{fontenc}
  \usepackage[utf8]{inputenc}
  \usepackage{textcomp} % provide euro and other symbols
\else % if luatex or xetex
  \usepackage{unicode-math}
  \defaultfontfeatures{Scale=MatchLowercase}
  \defaultfontfeatures[\rmfamily]{Ligatures=TeX,Scale=1}
\fi
% Use upquote if available, for straight quotes in verbatim environments
\IfFileExists{upquote.sty}{\usepackage{upquote}}{}
\IfFileExists{microtype.sty}{% use microtype if available
  \usepackage[]{microtype}
  \UseMicrotypeSet[protrusion]{basicmath} % disable protrusion for tt fonts
}{}
\makeatletter
\@ifundefined{KOMAClassName}{% if non-KOMA class
  \IfFileExists{parskip.sty}{%
    \usepackage{parskip}
  }{% else
    \setlength{\parindent}{0pt}
    \setlength{\parskip}{6pt plus 2pt minus 1pt}}
}{% if KOMA class
  \KOMAoptions{parskip=half}}
\makeatother
\usepackage{xcolor}
\IfFileExists{xurl.sty}{\usepackage{xurl}}{} % add URL line breaks if available
\IfFileExists{bookmark.sty}{\usepackage{bookmark}}{\usepackage{hyperref}}
\hypersetup{
  hidelinks,
  pdfcreator={LaTeX via pandoc}}
\urlstyle{same} % disable monospaced font for URLs
\usepackage[margin=1in]{geometry}
\usepackage{color}
\usepackage{fancyvrb}
\newcommand{\VerbBar}{|}
\newcommand{\VERB}{\Verb[commandchars=\\\{\}]}
\DefineVerbatimEnvironment{Highlighting}{Verbatim}{commandchars=\\\{\}}
% Add ',fontsize=\small' for more characters per line
\usepackage{framed}
\definecolor{shadecolor}{RGB}{248,248,248}
\newenvironment{Shaded}{\begin{snugshade}}{\end{snugshade}}
\newcommand{\AlertTok}[1]{\textcolor[rgb]{0.94,0.16,0.16}{#1}}
\newcommand{\AnnotationTok}[1]{\textcolor[rgb]{0.56,0.35,0.01}{\textbf{\textit{#1}}}}
\newcommand{\AttributeTok}[1]{\textcolor[rgb]{0.77,0.63,0.00}{#1}}
\newcommand{\BaseNTok}[1]{\textcolor[rgb]{0.00,0.00,0.81}{#1}}
\newcommand{\BuiltInTok}[1]{#1}
\newcommand{\CharTok}[1]{\textcolor[rgb]{0.31,0.60,0.02}{#1}}
\newcommand{\CommentTok}[1]{\textcolor[rgb]{0.56,0.35,0.01}{\textit{#1}}}
\newcommand{\CommentVarTok}[1]{\textcolor[rgb]{0.56,0.35,0.01}{\textbf{\textit{#1}}}}
\newcommand{\ConstantTok}[1]{\textcolor[rgb]{0.00,0.00,0.00}{#1}}
\newcommand{\ControlFlowTok}[1]{\textcolor[rgb]{0.13,0.29,0.53}{\textbf{#1}}}
\newcommand{\DataTypeTok}[1]{\textcolor[rgb]{0.13,0.29,0.53}{#1}}
\newcommand{\DecValTok}[1]{\textcolor[rgb]{0.00,0.00,0.81}{#1}}
\newcommand{\DocumentationTok}[1]{\textcolor[rgb]{0.56,0.35,0.01}{\textbf{\textit{#1}}}}
\newcommand{\ErrorTok}[1]{\textcolor[rgb]{0.64,0.00,0.00}{\textbf{#1}}}
\newcommand{\ExtensionTok}[1]{#1}
\newcommand{\FloatTok}[1]{\textcolor[rgb]{0.00,0.00,0.81}{#1}}
\newcommand{\FunctionTok}[1]{\textcolor[rgb]{0.00,0.00,0.00}{#1}}
\newcommand{\ImportTok}[1]{#1}
\newcommand{\InformationTok}[1]{\textcolor[rgb]{0.56,0.35,0.01}{\textbf{\textit{#1}}}}
\newcommand{\KeywordTok}[1]{\textcolor[rgb]{0.13,0.29,0.53}{\textbf{#1}}}
\newcommand{\NormalTok}[1]{#1}
\newcommand{\OperatorTok}[1]{\textcolor[rgb]{0.81,0.36,0.00}{\textbf{#1}}}
\newcommand{\OtherTok}[1]{\textcolor[rgb]{0.56,0.35,0.01}{#1}}
\newcommand{\PreprocessorTok}[1]{\textcolor[rgb]{0.56,0.35,0.01}{\textit{#1}}}
\newcommand{\RegionMarkerTok}[1]{#1}
\newcommand{\SpecialCharTok}[1]{\textcolor[rgb]{0.00,0.00,0.00}{#1}}
\newcommand{\SpecialStringTok}[1]{\textcolor[rgb]{0.31,0.60,0.02}{#1}}
\newcommand{\StringTok}[1]{\textcolor[rgb]{0.31,0.60,0.02}{#1}}
\newcommand{\VariableTok}[1]{\textcolor[rgb]{0.00,0.00,0.00}{#1}}
\newcommand{\VerbatimStringTok}[1]{\textcolor[rgb]{0.31,0.60,0.02}{#1}}
\newcommand{\WarningTok}[1]{\textcolor[rgb]{0.56,0.35,0.01}{\textbf{\textit{#1}}}}
\usepackage{longtable,booktabs}
% Correct order of tables after \paragraph or \subparagraph
\usepackage{etoolbox}
\makeatletter
\patchcmd\longtable{\par}{\if@noskipsec\mbox{}\fi\par}{}{}
\makeatother
% Allow footnotes in longtable head/foot
\IfFileExists{footnotehyper.sty}{\usepackage{footnotehyper}}{\usepackage{footnote}}
\makesavenoteenv{longtable}
\usepackage{graphicx,grffile}
\makeatletter
\def\maxwidth{\ifdim\Gin@nat@width>\linewidth\linewidth\else\Gin@nat@width\fi}
\def\maxheight{\ifdim\Gin@nat@height>\textheight\textheight\else\Gin@nat@height\fi}
\makeatother
% Scale images if necessary, so that they will not overflow the page
% margins by default, and it is still possible to overwrite the defaults
% using explicit options in \includegraphics[width, height, ...]{}
\setkeys{Gin}{width=\maxwidth,height=\maxheight,keepaspectratio}
% Set default figure placement to htbp
\makeatletter
\def\fps@figure{htbp}
\makeatother
\setlength{\emergencystretch}{3em} % prevent overfull lines
\providecommand{\tightlist}{%
  \setlength{\itemsep}{0pt}\setlength{\parskip}{0pt}}
\setcounter{secnumdepth}{-\maxdimen} % remove section numbering

\author{}
\date{\vspace{-2.5em}}

\begin{document}

:--- title: ``Projet d'Analyse et de Fouille de Données Massive''
output: html\_document: df\_print: paged pdf\_document: default ---

\hypertarget{projet-danalyse-et-de-fouille-de-donnuxe9es-massive}{%
\section{Projet d'Analyse et de Fouille de Données
Massive}\label{projet-danalyse-et-de-fouille-de-donnuxe9es-massive}}

\hypertarget{introduction}{%
\subsection{Introduction :}\label{introduction}}

 Le but du projet est de manipuler différentes méthodes d'analyses de
données sur un jeu de données constitué d'une centaine d'individus et
d'une dizaine de variables qualitatives ou quantitatives. Nous avons
choisi comme jeu de données les vidéos présentes dans la catégorie
``Tendances'' du site YouTube France sur une période allant de novembre
2017 à juin 2018. La catégorie ``Tendances'' de YouTube met en évidence
les vidéos les plus vues et les plus appréciées par les utilisateurs de
la plateforme. Elle permet donc une plus grande visibilité pour les
créateurs de contenu. Une vidéo est placée dans la catégorie
``Tendances'' par un algorithme se fondant sur les statistiques de la
vidéo ainsi que sur les interactions des utilisateurs avec celle-ci.
L'analyse des facteurs permettant à une vidéo d'être présente en
``Tendances'' peut donc être intéressante pour les vidéastes de la
plateforme souhaitant être mis en avant.

\hypertarget{pruxe9sentation-de-nos-donnuxe9es}{%
\subsubsection{Présentation de nos données
:}\label{pruxe9sentation-de-nos-donnuxe9es}}

 Nous avons extrait nos données du site Kaggle
(\url{https://www.kaggle.com/datasnaek/youtube-new}). Ces données
contiennent un ensemble de statistiques sur les vidéos de la catégorie
``Tendances'' de YouTube. Nous avons décidé d'extraire les statistiques
qui nous paraissaient pertinentes pour pouvoir déterminer les facteurs
influant sur la présence ou non d'une vidéo dans la catégorie
``Tendances''. Ces statistiques sont resumées dans le tableau suivant et
seront utilisées comme variables qualitatives ou quantitatives dans les
différentes méthodes d'analyse des données.

\begin{longtable}[]{@{}cc@{}}
\toprule
Nom & Attribut\tabularnewline
\midrule
\endhead
Taux LOWER Case & quantitatif\tabularnewline
Taux UPPER Case & quantitatif\tabularnewline
Nb tags & quantitatif\tabularnewline
Nb vues & quantitatif\tabularnewline
Nb commentaires & quantitatif\tabularnewline
Nb likes & quantitatif\tabularnewline
Nb dislikes & quantitatif\tabularnewline
Taille de la description (en mots) & quantitatif\tabularnewline
Catégorie & qualitatif\tabularnewline
Jour de la semaine & qualitatif\tabularnewline
Moment de la journée & qualitatif\tabularnewline
Nombre de lien en description & quantitatif\tabularnewline
\bottomrule
\end{longtable}

 Nous pouvons nous attendre à une forte corrélation entre la présence
d'une vidéo en ``Tendances'' et son nombre de vues, de mentions
``J'aime'' et de commentaires. Les vidéos sont placées dans la catégorie
``Tendances'' principalement à cause du fait que les utilisateurs ont
beaucoup interragi avec. Néanmoins, d'autres variables peuvent influer
de manière indirecte sur les paramètres cités précedemment.

 Le nombre de majuscules nous semble pertinent, car nous avons remarqué
une tendance des vidéastes à utiliser des titres uniquement composés de
majuscules afin d'attirer l'oeil des utilisateurs de YouTube, et donc
d'augmenter plus facilement le nombre de ``vues'' sur la vidéo.

 Le nombre de tags est un autre paramètre important car YouTube propose
des recommandations personnalisées aux utilisateurs en fonction de leurs
centres d'intérêts. Une vidéo pourrait donc bénéficier d'une audience
plus importante si elle était recommandée à plusieurs publics
différents. De plus, certains tags sont considérés comme ``tendance''
lorsqu'ils sont en lien avec l'actualité. Ainsi, lors des incidents
récents entre l'Iran et les Etats-Unis, le tag ``WWIII'' est devenu
fréquent sur la plateforme car de nombreux internautes craignaient une
détérioration de la situation. Cela a permis à certains créateurs de
contenu politique, par exemple, d'exploiter cette tendance en publiant
du contenu connexe à ce tag.

 Un autre élément qui pourrait nous donner des indicateurs sur la mise
en avant d'une vidéo par l'algorithme de YouTube serait l'espace de
description de la vidéo. Chaque vidéo doit posséder une description. Les
descriptions sont généralement utilisées pour résumer la vidéo, mettre à
disposition des utilisateurs les liens vers les sources de la vidéo, les
différentes pages sur les réseaux sociaux du créateur mais aussi vers
des magasins en ligne, des sites de sponsors ou alors d'autres chaînes
YouTube dans le cadre d'une collaboration. Nous allons plus
particulièrement nous intéresser à la longueur de la description. En
effet, la majorité des créateurs qui vivent de leur activité sur YouTube
possèdent des descriptions longues et détaillées. Cela peut bien sûr
aussi être le cas pour des vidéastes plus amateurs. Nous avons également
compté le nombre de liens que possédait la description. En effet, mis à
part les chaînes d'information et de vulgarisation, la majorité des
chaînes proposent un lien vers un magasin en ligne, trois ou quatre
liens vers des pages de réseaux sociaux et éventuellement des liens vers
des sites de sponsors. Cependant, lorsqu'une vidéo est le fruit d'une
collaboration avec d'autres créateurs, un lien vers la chaîne YouTube de
cette autre personne est généralement présent dans la description. Les
vidéos réalisées avec plusieurs vidéastes attirent un public plus large
et sont devenues beaucoup plus communes qu'auparavant sur la plateforme.
Nous esperons donc, à travers cette variable, mettre en avant ce
phénomène.

 La publication d'une vidéo peut également être ciblée afin de toucher
le plus de personnes possible (une vidéo publiée le vendredi soir à 18h
aura probablement plus d'impact qu'une vidéo publiée le mardi matin à
8h). Nous avons donc choisi de nous intéresser au moment auquel une
vidéo a été mise en ligne. Nous avons découpé de la manière suivante les
moments de la journée pour nos analyses : 00:00 -\textgreater{} 11:59 =
``morning'', 12:00 -\textgreater{} 19:00 = ``afternoon'' et le reste est
considéré comme ``evening''.

 Nous serons également en mesure d'exhiber les types de vidéos qui sont
le plus visionnées par le public français, à travers la catégorie de la
vidéo. Ce facteur nous permettra également de comparer les audiences des
vidéos appartenant à différentes catégories (par exemple, les personnes
regardant des vidéos sportives seraient peut-être plus enclines à
consulter des vidéos de mode ou de divertissement que les personnes
visionnant des vidéos ``Gaming'').

\hypertarget{traitement-du-jeu-de-donnuxe9es}{%
\subsubsection{Traitement du jeu de
données}\label{traitement-du-jeu-de-donnuxe9es}}

 A partir du jeu de données télechargé au format csv, nous avons généré
un nouveau fichier csv contenant, pour chaque vidéo, les statistiques
listées dans le tableau précédent Vous pouvez retrouver le code commenté
pour l'extraction et le traitement des données brutes initiales dans le
fichier `processData.py'.

\hypertarget{analyse-des-donnuxe9es}{%
\subsubsection{Analyse des données}\label{analyse-des-donnuxe9es}}

 Dans cette partie, nous allons analyser les données que nous avons
extraites et qui se trouvent dans le fichier ``youtubeTrends.csv''. Ce
fichier contient 40703 individus. Néanmoins les méthodes d'analyse
détaillées dans cette partie ne porteront que sur une centaine
d'invidus. Nous sélectionnons donc 100 vidéos de manière aléatoire dans
le fichier ``youtubeTrends.csv''.

Code pour selectionner les 100 individus aléatoirement :

\begin{Shaded}
\begin{Highlighting}[]
\CommentTok{# data = read.csv("./DATA/youtubeTrends.csv")}
\CommentTok{# data = data[sample(nrow(data), 100),]}
\end{Highlighting}
\end{Shaded}

Code pour récupérer les 100 individus que nous avons utilisé et
sauvegardé :

\begin{Shaded}
\begin{Highlighting}[]
\KeywordTok{load}\NormalTok{(}\StringTok{'./random_data.rdata'}\NormalTok{)}
\end{Highlighting}
\end{Shaded}

\hypertarget{analyse-en-composantes-principales-acp}{%
\section{Analyse en Composantes Principales
(ACP)}\label{analyse-en-composantes-principales-acp}}

 La première analyse que nous allons effectuer sera une ACP. Cette
analyse ne porte que sur des données qualitatives. Ainsi, il faut
retirer de nos données les variables qualitatives. Ensuite, on applique
l'ACP sur les données filtrées.

\begin{Shaded}
\begin{Highlighting}[]
\CommentTok{# On retire les données qualitatives}
\NormalTok{data_ACP =}\StringTok{ }\NormalTok{data}
\NormalTok{data_ACP}\OperatorTok{$}\NormalTok{category <-}\StringTok{ }\OtherTok{NULL}
\NormalTok{data_ACP}\OperatorTok{$}\NormalTok{momentOfDay <-}\StringTok{ }\OtherTok{NULL}
\NormalTok{data_ACP}\OperatorTok{$}\NormalTok{day <-}\StringTok{ }\OtherTok{NULL}
\NormalTok{data_ACP}\OperatorTok{$}\NormalTok{index <-}\StringTok{ }\OtherTok{NULL}

\CommentTok{# On retire les données considérées comme faiblement corellées après les premiers tests }
\NormalTok{data_ACP}\OperatorTok{$}\NormalTok{nbTags <-}\StringTok{ }\OtherTok{NULL}
\NormalTok{data_ACP}\OperatorTok{$}\NormalTok{nbWords <-}\StringTok{ }\OtherTok{NULL}
\NormalTok{data_ACP}\OperatorTok{$}\NormalTok{nbLinks <-}\StringTok{ }\OtherTok{NULL}

\CommentTok{# On applique l'ACP sur les données filtrées}
\NormalTok{res.pca =}\StringTok{ }\KeywordTok{PCA}\NormalTok{(data_ACP, }\DataTypeTok{scale.unit=}\OtherTok{TRUE}\NormalTok{, }\DataTypeTok{ncp=}\DecValTok{6}\NormalTok{, }\DataTypeTok{graph=}\NormalTok{T) }
\end{Highlighting}
\end{Shaded}

\includegraphics{projet_files/figure-latex/unnamed-chunk-3-1.pdf}
\includegraphics{projet_files/figure-latex/unnamed-chunk-3-2.pdf}

 Sur la figure, on constate que les données sont bien représentées. En
effet, 46.89\% des données sont représentées sur l'axe 1 et 28.47\% sur
l'axe 2. On voit que le nombre de likes et dislikes sont corrélés.
Néanmoins, après analyse, il apparaît que ces deux variables ne
capturent pas vraiment ce que l'on voulait. En effet, nous ne savons pas
en quoi elles sont corrélées (si un grand nombre de likes est associé à
un grand nombre de dislikes\ldots). Ainsi, pour voir l'impact du nombre
de likes et de dislikes, on introduit deux nouvelles variables
quantitatives : le ratio de like et le ratio de dislikes par vidéo.

\begin{Shaded}
\begin{Highlighting}[]
\CommentTok{# On retire les données qualitatives}
\NormalTok{data_ACP =}\StringTok{ }\NormalTok{data}
\NormalTok{data_ACP}\OperatorTok{$}\NormalTok{category <-}\StringTok{ }\OtherTok{NULL}
\NormalTok{data_ACP}\OperatorTok{$}\NormalTok{momentOfDay <-}\StringTok{ }\OtherTok{NULL}
\NormalTok{data_ACP}\OperatorTok{$}\NormalTok{day <-}\StringTok{ }\OtherTok{NULL}
\NormalTok{data_ACP}\OperatorTok{$}\NormalTok{index <-}\StringTok{ }\OtherTok{NULL}

\CommentTok{# On retire les données faiblement corellé}
\NormalTok{data_ACP}\OperatorTok{$}\NormalTok{nbTags <-}\StringTok{ }\OtherTok{NULL}
\NormalTok{data_ACP}\OperatorTok{$}\NormalTok{nbWords <-}\StringTok{ }\OtherTok{NULL}
\NormalTok{data_ACP}\OperatorTok{$}\NormalTok{nbLinks <-}\StringTok{ }\OtherTok{NULL}

\CommentTok{# Ajout du ratio de like et dislike}
\NormalTok{data_ACP}\OperatorTok{$}\NormalTok{ratioLikes <-}\StringTok{ }\NormalTok{data_ACP}\OperatorTok{$}\NormalTok{nbLikes}\OperatorTok{/}\NormalTok{(data_ACP}\OperatorTok{$}\NormalTok{nbLikes }\OperatorTok{+}\StringTok{ }\NormalTok{data_ACP}\OperatorTok{$}\NormalTok{nbDislikes)}
\NormalTok{data_ACP}\OperatorTok{$}\NormalTok{ratioDislikes <-}\StringTok{ }\NormalTok{data_ACP}\OperatorTok{$}\NormalTok{nbDislikes}\OperatorTok{/}\NormalTok{(data_ACP}\OperatorTok{$}\NormalTok{nbLikes }\OperatorTok{+}\StringTok{ }\NormalTok{data_ACP}\OperatorTok{$}\NormalTok{nbDislikes)}

\NormalTok{data_ACP}\OperatorTok{$}\NormalTok{nbLikes <-}\StringTok{ }\OtherTok{NULL}
\NormalTok{data_ACP}\OperatorTok{$}\NormalTok{nbDislikes <-}\StringTok{ }\OtherTok{NULL}

\NormalTok{res.pca =}\StringTok{ }\KeywordTok{PCA}\NormalTok{(data_ACP, }\DataTypeTok{scale.unit=}\OtherTok{TRUE}\NormalTok{, }\DataTypeTok{ncp=}\DecValTok{6}\NormalTok{, }\DataTypeTok{graph=}\NormalTok{T) }
\end{Highlighting}
\end{Shaded}

\begin{verbatim}
## Warning in PCA(data_ACP, scale.unit = TRUE, ncp = 6, graph = T): Missing values
## are imputed by the mean of the variable: you should use the imputePCA function
## of the missMDA package
\end{verbatim}

\includegraphics{projet_files/figure-latex/unnamed-chunk-4-1.pdf}
\includegraphics{projet_files/figure-latex/unnamed-chunk-4-2.pdf}   On
observe que le ratio de likes et de dislikes sont inversement corrélés.
En effet, cela semble cohérent qu'une vidéo en ``Tendances'' ayant un
fort nombre de mentions ``J'aime'' ait une faible proportion de mentions
``Je n'aime pas'' et c'est bien ce que l'on constate sur le graphe
résultant de l'ACP.

\hypertarget{analyse-des-composantes-multiples-acm}{%
\subsection{Analyse des Composantes Multiples
(ACM)}\label{analyse-des-composantes-multiples-acm}}

 Dans cette partie, nous allons réaliser une ACM sur 3 attributs
qualitatifs : le jour de la semaine où la vidéo est publiée, le moment
de la journée et la catégorie de la vidéo.

 Dans un premier temps, nous préparons les données en enlevant la moitié
des données. En effet, si l'on choisit d'effectuer l'ACP sur l'ensemble
des données, le graphe résultant contient trop de points superposés et
il est difficile de l'interpréter. Ensuite, il faut effectuer une petite
opération pour transformer nos données en type Factor (un type
spécifique à R). Cette opération est necessaire car sans cela, nous
obtenons une erreur. Enfin, on retire tous les attributs quantitatifs
sauf le nombre de likes, dislikes et de commentaires. On conserve ces
attributs car on souhaite regrouper les attributs qualitatifs en
fonction de leur poids sur le fait que la vidéo soit en ``Tendances'',
qui se représente à l'aide des 3 variables quantitatives ci-dessus.

\begin{Shaded}
\begin{Highlighting}[]
\NormalTok{data_ACM =}\StringTok{ }\NormalTok{data[}\DecValTok{1}\OperatorTok{:}\DecValTok{50}\NormalTok{,] }
\NormalTok{i=}\DecValTok{0}
\ControlFlowTok{while}\NormalTok{(i}\OperatorTok{<}\KeywordTok{ncol}\NormalTok{(data_ACM))\{}
\NormalTok{  i=i}\OperatorTok{+}\DecValTok{1}
\NormalTok{  data_ACM[,i]=}\KeywordTok{as.factor}\NormalTok{(data_ACM[,i])}
\NormalTok{\}}
\NormalTok{data_ACM}\OperatorTok{$}\NormalTok{index <-}\StringTok{ }\OtherTok{NULL}
\NormalTok{data_ACM}\OperatorTok{$}\NormalTok{nbTags <-}\StringTok{ }\OtherTok{NULL}
\NormalTok{data_ACM}\OperatorTok{$}\NormalTok{nbWords <-}\StringTok{ }\OtherTok{NULL}
\NormalTok{data_ACM}\OperatorTok{$}\NormalTok{nbLinks <-}\StringTok{ }\OtherTok{NULL}
\NormalTok{data_ACM}\OperatorTok{$}\NormalTok{X.lowerCase <-}\StringTok{ }\OtherTok{NULL}
\NormalTok{data_ACM}\OperatorTok{$}\NormalTok{X.upperCase <-}\StringTok{ }\OtherTok{NULL}
\end{Highlighting}
\end{Shaded}

\hypertarget{moment-de-la-semaine}{%
\subsubsection{Moment de la semaine}\label{moment-de-la-semaine}}

 Nous nous interessons, dans un premier temps, au moment de la journée
où la vidéo à été publiée. On s'attend éventuellement à voir regroupés
ensemble des jours de la semaine qui sont proches (en fonction de s'ils
sont en début, milieu et fin de semaine).

\begin{Shaded}
\begin{Highlighting}[]
\NormalTok{res.mca_}\DecValTok{1}\NormalTok{ =}\StringTok{ }\KeywordTok{MCA}\NormalTok{(data_ACM, }\DataTypeTok{quali.sup =} \KeywordTok{c}\NormalTok{(}\DecValTok{4}\NormalTok{), }\DataTypeTok{graph =} \OtherTok{FALSE}\NormalTok{)}
\KeywordTok{plot.MCA}\NormalTok{(res.mca_}\DecValTok{1}\NormalTok{, }\DataTypeTok{invisible=}\KeywordTok{c}\NormalTok{(}\StringTok{"var"}\NormalTok{), }\DataTypeTok{cex=}\FloatTok{0.75}\NormalTok{)}
\end{Highlighting}
\end{Shaded}

\includegraphics{projet_files/figure-latex/unnamed-chunk-6-1.pdf}

 Les résultats obtenus sont plutôt satisfaisants, même si l'on
représente peu les données (2.82\%). On peut observer que certains jours
sont reliés comme le Lundi, Mardi et Mercredi. Cela semble cohérent
qu'ils aient la même visibilité, car ils se trouvent tous deux en milieu
de semaine. De la même manière, le Vendredi est extrèmement corrélé au
Samedi et au Dimanche. Résultat plus étonnant, on constate que le Jeudi
est isolé dans le graphique, les vidéos publiées le Jeudi doivent donc
être visualisées par un public ciblé.

\hypertarget{moment-de-la-journuxe9e}{%
\subsubsection{Moment de la journée}\label{moment-de-la-journuxe9e}}

 La prochaine ACM cherche à étudier les corrélations entre les moments
de la journée où sont publiés une vidéo. Intuitivement, on pourrait
penser que les vidéos qui sortent l'après-midi n'ont pas le même
``succès'' qu'une vidéo qui a été publiée le matin ou le soir.

\begin{Shaded}
\begin{Highlighting}[]
\NormalTok{res.mca_}\DecValTok{2}\NormalTok{ =}\StringTok{ }\KeywordTok{MCA}\NormalTok{(data_ACM, }\DataTypeTok{quali.sup =} \KeywordTok{c}\NormalTok{(}\DecValTok{5}\NormalTok{), }\DataTypeTok{graph =} \OtherTok{FALSE}\NormalTok{)}
\KeywordTok{plot.MCA}\NormalTok{(res.mca_}\DecValTok{2}\NormalTok{, }\DataTypeTok{invisible=}\KeywordTok{c}\NormalTok{(}\StringTok{"var"}\NormalTok{), }\DataTypeTok{cex=}\FloatTok{0.75}\NormalTok{)}
\end{Highlighting}
\end{Shaded}

\includegraphics{projet_files/figure-latex/unnamed-chunk-7-1.pdf}

 Sur le graphique résultant, on constate que les individus peuvent être
regroupés en deux catégories. Tout d'abord, la majorité des vidéos en
``Tendances'' sont publiées l'après-midi ou le soir. Néanmoins, un
certain nombre de vidéos présentes dans la catégorie ``Tendances'' ont
été mises en ligne le matin. Cela peut s'interpréter de deux manières.
D'une part, ces vidéos peuvent être destinées à un public particulier,
pouvant les regarder le matin. D'autre part, ces vidéos peuvent être
brèves (comme celles présentes sur les réseaux sociaux) donc pouvant
être visionnées rapidement par les utilisateurs et/ou traitant
d'actualité.

\hypertarget{catuxe9gorie}{%
\subsubsection{Catégorie}\label{catuxe9gorie}}

  Enfin, la dernière ACM porte sur les catégories des vidéos présentes
dans la catégorie ``Tendances''.

\begin{Shaded}
\begin{Highlighting}[]
\NormalTok{res.mca_}\DecValTok{3}\NormalTok{ =}\StringTok{ }\KeywordTok{MCA}\NormalTok{(data_ACM, }\DataTypeTok{quali.sup =} \KeywordTok{c}\NormalTok{(}\DecValTok{7}\NormalTok{), }\DataTypeTok{graph =} \OtherTok{FALSE}\NormalTok{)}
\KeywordTok{plot.MCA}\NormalTok{(res.mca_}\DecValTok{3}\NormalTok{, }\DataTypeTok{invisible=}\KeywordTok{c}\NormalTok{(}\StringTok{"var"}\NormalTok{), }\DataTypeTok{cex=}\FloatTok{0.75}\NormalTok{)}
\end{Highlighting}
\end{Shaded}

\includegraphics{projet_files/figure-latex/unnamed-chunk-8-1.pdf}  Nous
remarquons que de nombreuses catgéories sont liées entre elles dans le
sens où elles ont des audiences communes. Ainsi, sur le graphique, nous
constatons que les personnes regardant des vidéos de la catégorie
``Howto \& Style'', visionnent également des vidéos appartenant aux
catégories ``News and Politics'' et ``Comedy''. De manière similaire,
les personnes regardant des vidéos de la catégorie ``Gaming'' visionnent
également des vidéos traitant de sport et de voyage.

\hypertarget{analyse-factorielle-des-correspondances-afc}{%
\subsection{Analyse Factorielle des Correspondances
(AFC)}\label{analyse-factorielle-des-correspondances-afc}}

 La dernière analyse que nous proposons est l'AFC. Grâce à cette
méthode, nous allons mettre en évidence les attributs qui fédèrent le
plus d'individus.

\begin{Shaded}
\begin{Highlighting}[]
\NormalTok{data_AFC =}\StringTok{ }\NormalTok{data}
\NormalTok{res.afc =}\StringTok{ }\KeywordTok{CA}\NormalTok{(data_AFC[, }\KeywordTok{c}\NormalTok{(}\DecValTok{2}\OperatorTok{:}\DecValTok{5}\NormalTok{ ,}\DecValTok{12}\OperatorTok{:}\DecValTok{13}\NormalTok{)])}
\end{Highlighting}
\end{Shaded}

\includegraphics{projet_files/figure-latex/unnamed-chunk-9-1.pdf}

La figure obtenue est très intéressante, tout d'abord on constate que
toutes les vidéos en tendance partagent un nombre de vues relativement
proche. Le nombre de liens en description de la vidéo est également
proche du nuage de points. Cela confirme que les vidéos collaboratives
entre créateurs génèrent plus de visionnages. Un autre attribut proche
du nombre de liens est le nombre de tags de la vidéo. Les tags aident à
mettre la vidéo en avant et sont utiles, notamment, si la vidéo traite
de l'actualité. De plus, certains tags peuvent momentanément profiter
d'une certaine notoriété. A la sortie d'un nouveau jeu vidéo, il est
probable que le tag ``Jeux'' ou ``Gaming'' soit mis en avant. Enfin, on
peut remarquer que l'attribut ``UpperCase'' est plus proche du nuage de
points que l'attribut ``LowerCase''. Cela semble indiquer qu'écrire les
titres en majuscule attire davantage les utilisateurs du site.

\hypertarget{conclusion}{%
\subsubsection{Conclusion}\label{conclusion}}

Nous avons analysé les données issues des vidéos de la catégorie
``Tendances'' du site YouTube (France). Ce travail nous a permis
d'appliquer les méthodes d'analyse et de fouille de données que nous
avons vues en cours. Nous avons ainsi pu confirmer certaines hypothèses
sur les facteurs aboutissant au fait qu'une vidéo soit présente en
``Tendances'' mais aussi découvrir des liens qui nous ont étonnés.

\end{document}
