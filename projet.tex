% Options for packages loaded elsewhere
\PassOptionsToPackage{unicode}{hyperref}
\PassOptionsToPackage{hyphens}{url}
%
\documentclass[
]{article}
\usepackage{lmodern}
\usepackage{amssymb,amsmath}
\usepackage{ifxetex,ifluatex}
\ifnum 0\ifxetex 1\fi\ifluatex 1\fi=0 % if pdftex
  \usepackage[T1]{fontenc}
  \usepackage[utf8]{inputenc}
  \usepackage{textcomp} % provide euro and other symbols
\else % if luatex or xetex
  \usepackage{unicode-math}
  \defaultfontfeatures{Scale=MatchLowercase}
  \defaultfontfeatures[\rmfamily]{Ligatures=TeX,Scale=1}
\fi
% Use upquote if available, for straight quotes in verbatim environments
\IfFileExists{upquote.sty}{\usepackage{upquote}}{}
\IfFileExists{microtype.sty}{% use microtype if available
  \usepackage[]{microtype}
  \UseMicrotypeSet[protrusion]{basicmath} % disable protrusion for tt fonts
}{}
\makeatletter
\@ifundefined{KOMAClassName}{% if non-KOMA class
  \IfFileExists{parskip.sty}{%
    \usepackage{parskip}
  }{% else
    \setlength{\parindent}{0pt}
    \setlength{\parskip}{6pt plus 2pt minus 1pt}}
}{% if KOMA class
  \KOMAoptions{parskip=half}}
\makeatother
\usepackage{xcolor}
\IfFileExists{xurl.sty}{\usepackage{xurl}}{} % add URL line breaks if available
\IfFileExists{bookmark.sty}{\usepackage{bookmark}}{\usepackage{hyperref}}
\hypersetup{
  pdftitle={Projet},
  hidelinks,
  pdfcreator={LaTeX via pandoc}}
\urlstyle{same} % disable monospaced font for URLs
\usepackage[margin=1in]{geometry}
\usepackage{color}
\usepackage{fancyvrb}
\newcommand{\VerbBar}{|}
\newcommand{\VERB}{\Verb[commandchars=\\\{\}]}
\DefineVerbatimEnvironment{Highlighting}{Verbatim}{commandchars=\\\{\}}
% Add ',fontsize=\small' for more characters per line
\usepackage{framed}
\definecolor{shadecolor}{RGB}{248,248,248}
\newenvironment{Shaded}{\begin{snugshade}}{\end{snugshade}}
\newcommand{\AlertTok}[1]{\textcolor[rgb]{0.94,0.16,0.16}{#1}}
\newcommand{\AnnotationTok}[1]{\textcolor[rgb]{0.56,0.35,0.01}{\textbf{\textit{#1}}}}
\newcommand{\AttributeTok}[1]{\textcolor[rgb]{0.77,0.63,0.00}{#1}}
\newcommand{\BaseNTok}[1]{\textcolor[rgb]{0.00,0.00,0.81}{#1}}
\newcommand{\BuiltInTok}[1]{#1}
\newcommand{\CharTok}[1]{\textcolor[rgb]{0.31,0.60,0.02}{#1}}
\newcommand{\CommentTok}[1]{\textcolor[rgb]{0.56,0.35,0.01}{\textit{#1}}}
\newcommand{\CommentVarTok}[1]{\textcolor[rgb]{0.56,0.35,0.01}{\textbf{\textit{#1}}}}
\newcommand{\ConstantTok}[1]{\textcolor[rgb]{0.00,0.00,0.00}{#1}}
\newcommand{\ControlFlowTok}[1]{\textcolor[rgb]{0.13,0.29,0.53}{\textbf{#1}}}
\newcommand{\DataTypeTok}[1]{\textcolor[rgb]{0.13,0.29,0.53}{#1}}
\newcommand{\DecValTok}[1]{\textcolor[rgb]{0.00,0.00,0.81}{#1}}
\newcommand{\DocumentationTok}[1]{\textcolor[rgb]{0.56,0.35,0.01}{\textbf{\textit{#1}}}}
\newcommand{\ErrorTok}[1]{\textcolor[rgb]{0.64,0.00,0.00}{\textbf{#1}}}
\newcommand{\ExtensionTok}[1]{#1}
\newcommand{\FloatTok}[1]{\textcolor[rgb]{0.00,0.00,0.81}{#1}}
\newcommand{\FunctionTok}[1]{\textcolor[rgb]{0.00,0.00,0.00}{#1}}
\newcommand{\ImportTok}[1]{#1}
\newcommand{\InformationTok}[1]{\textcolor[rgb]{0.56,0.35,0.01}{\textbf{\textit{#1}}}}
\newcommand{\KeywordTok}[1]{\textcolor[rgb]{0.13,0.29,0.53}{\textbf{#1}}}
\newcommand{\NormalTok}[1]{#1}
\newcommand{\OperatorTok}[1]{\textcolor[rgb]{0.81,0.36,0.00}{\textbf{#1}}}
\newcommand{\OtherTok}[1]{\textcolor[rgb]{0.56,0.35,0.01}{#1}}
\newcommand{\PreprocessorTok}[1]{\textcolor[rgb]{0.56,0.35,0.01}{\textit{#1}}}
\newcommand{\RegionMarkerTok}[1]{#1}
\newcommand{\SpecialCharTok}[1]{\textcolor[rgb]{0.00,0.00,0.00}{#1}}
\newcommand{\SpecialStringTok}[1]{\textcolor[rgb]{0.31,0.60,0.02}{#1}}
\newcommand{\StringTok}[1]{\textcolor[rgb]{0.31,0.60,0.02}{#1}}
\newcommand{\VariableTok}[1]{\textcolor[rgb]{0.00,0.00,0.00}{#1}}
\newcommand{\VerbatimStringTok}[1]{\textcolor[rgb]{0.31,0.60,0.02}{#1}}
\newcommand{\WarningTok}[1]{\textcolor[rgb]{0.56,0.35,0.01}{\textbf{\textit{#1}}}}
\usepackage{longtable,booktabs}
% Correct order of tables after \paragraph or \subparagraph
\usepackage{etoolbox}
\makeatletter
\patchcmd\longtable{\par}{\if@noskipsec\mbox{}\fi\par}{}{}
\makeatother
% Allow footnotes in longtable head/foot
\IfFileExists{footnotehyper.sty}{\usepackage{footnotehyper}}{\usepackage{footnote}}
\makesavenoteenv{longtable}
\usepackage{graphicx,grffile}
\makeatletter
\def\maxwidth{\ifdim\Gin@nat@width>\linewidth\linewidth\else\Gin@nat@width\fi}
\def\maxheight{\ifdim\Gin@nat@height>\textheight\textheight\else\Gin@nat@height\fi}
\makeatother
% Scale images if necessary, so that they will not overflow the page
% margins by default, and it is still possible to overwrite the defaults
% using explicit options in \includegraphics[width, height, ...]{}
\setkeys{Gin}{width=\maxwidth,height=\maxheight,keepaspectratio}
% Set default figure placement to htbp
\makeatletter
\def\fps@figure{htbp}
\makeatother
\setlength{\emergencystretch}{3em} % prevent overfull lines
\providecommand{\tightlist}{%
  \setlength{\itemsep}{0pt}\setlength{\parskip}{0pt}}
\setcounter{secnumdepth}{-\maxdimen} % remove section numbering

\title{Projet}
\author{}
\date{\vspace{-2.5em}}

\begin{document}
\maketitle

\hypertarget{projet-danalyse-et-de-fouille-de-donnuxe9es-massive}{%
\section{Projet d'Analyse et de Fouille de Données
Massive}\label{projet-danalyse-et-de-fouille-de-donnuxe9es-massive}}

doc si perdu : \url{http://rmarkdown.rstudio.com}

\hypertarget{introduction}{%
\subsection{Introduction :}\label{introduction}}

Le sujet nous demande de faire blabla

\hypertarget{pruxe9sentation-de-nos-donnuxe9es}{%
\subsection{Présentation de nos données
:}\label{pruxe9sentation-de-nos-donnuxe9es}}

Nous avons extrait nos donnée de
\url{https://www.kaggle.com/datasnaek/youtube-new}. Nous avons décidé
d'extraire les attributs suivant :

\begin{longtable}[]{@{}cc@{}}
\toprule
Nom & Attribut\tabularnewline
\midrule
\endhead
Taux LOWER Case & quantitatif\tabularnewline
Taux UPPER Case & quantitatif\tabularnewline
Nb tags & quantitatif\tabularnewline
Nb vues & quantitatif\tabularnewline
Nb commentaires & quantitatif\tabularnewline
Nb likes & quantitatif\tabularnewline
Nb dislikes & quantitatif\tabularnewline
Taille de la description (en mot) & quantitatif\tabularnewline
Catégorie & qualitatif\tabularnewline
Jour de la semaine & qualitatif\tabularnewline
Moment de la journée & qualitatif\tabularnewline
Nombre de lien en description & quantitatif\tabularnewline
\bottomrule
\end{longtable}

===== AJOUTER COMMENTAIRE SUR CE QU'ON ATTENDS DE CHAQUE VARIABLE =====

Vous pouvez retrouver le code commenté pour l'extraction dans le script
`processData.py'. Pour l'utiliser veuillez télécharger les données dans
un fichier `DATA' à la racine du projet. Vous n'avez besoin de ne
télécharger que ``FR\_category\_id.json'' et ``FRvideos.csv''.

\hypertarget{analyse-des-donnuxe9es}{%
\subsection{Analyse des données}\label{analyse-des-donnuxe9es}}

Dans cette partie, nous allons analyser les données que nous avons
extrait et qui se trouvent dans le fichier ``youtubeTrends.csv''. Ce
fichier contient 40703 individus, ce qui est beaucoups trop ! Nous
voulons avoir uniquement une centaine d'invidus, nous choisisons donc
une centaine d'individu aléatoirement.

Code pour récupérer les 100 individus que nous avons utilisé et
sauvegardé :

\begin{Shaded}
\begin{Highlighting}[]
\KeywordTok{setwd}\NormalTok{(}\StringTok{"/home/jyra/Documents/Enssat/3A/AFDM/DataMiningProject"}\NormalTok{)}
\KeywordTok{load}\NormalTok{(}\StringTok{'./random_data.rdata'}\NormalTok{)}
\end{Highlighting}
\end{Shaded}

\hypertarget{acp}{%
\subsubsection{ACP}\label{acp}}

On récupère les données :

La première analyse que nous allons essayé sera une ACP, pour ce faire,
il faut que l'on prépare nos données en retirant les données
qualitatives que nous avons. Ensuite On execute l'ACP sur nos données.

\begin{Shaded}
\begin{Highlighting}[]
\CommentTok{# On retire les données qualitatives}
\NormalTok{data_ACP =}\StringTok{ }\NormalTok{data}
\NormalTok{data_ACP}\OperatorTok{$}\NormalTok{category <-}\StringTok{ }\OtherTok{NULL}
\NormalTok{data_ACP}\OperatorTok{$}\NormalTok{momentOfDay <-}\StringTok{ }\OtherTok{NULL}
\NormalTok{data_ACP}\OperatorTok{$}\NormalTok{day <-}\StringTok{ }\OtherTok{NULL}
\NormalTok{data_ACP}\OperatorTok{$}\NormalTok{index <-}\StringTok{ }\OtherTok{NULL}

\CommentTok{# On retire les données considéré comme faiblement corellé après les premiers tests }
\NormalTok{data_ACP}\OperatorTok{$}\NormalTok{nbTags <-}\StringTok{ }\OtherTok{NULL}
\NormalTok{data_ACP}\OperatorTok{$}\NormalTok{nbWords <-}\StringTok{ }\OtherTok{NULL}
\NormalTok{data_ACP}\OperatorTok{$}\NormalTok{nbLinks <-}\StringTok{ }\OtherTok{NULL}

\NormalTok{res.pca =}\StringTok{ }\KeywordTok{PCA}\NormalTok{(data_ACP, }\DataTypeTok{scale.unit=}\OtherTok{TRUE}\NormalTok{, }\DataTypeTok{ncp=}\DecValTok{6}\NormalTok{, }\DataTypeTok{graph=}\NormalTok{T)}
\end{Highlighting}
\end{Shaded}

\includegraphics{projet_files/figure-latex/unnamed-chunk-2-1.pdf}
\includegraphics{projet_files/figure-latex/unnamed-chunk-2-2.pdf}

Sur la figure on peut voir que l'on représente bien les données X\% sur
l'axe 1 et Y\% sur l'axe 2. On voit que le nombre de like et dislike
sont corrélé, après analyse, on se rend compte que ces deux variables ne
capturent pas vraiment ce que l'on voulait. En effet

\begin{Shaded}
\begin{Highlighting}[]
\CommentTok{# On retire les données qualitatives}
\NormalTok{data_ACP =}\StringTok{ }\NormalTok{data}
\NormalTok{data_ACP}\OperatorTok{$}\NormalTok{category <-}\StringTok{ }\OtherTok{NULL}
\NormalTok{data_ACP}\OperatorTok{$}\NormalTok{momentOfDay <-}\StringTok{ }\OtherTok{NULL}
\NormalTok{data_ACP}\OperatorTok{$}\NormalTok{day <-}\StringTok{ }\OtherTok{NULL}
\NormalTok{data_ACP}\OperatorTok{$}\NormalTok{index <-}\StringTok{ }\OtherTok{NULL}

\CommentTok{# On retire les données faiblement corellé}
\NormalTok{data_ACP}\OperatorTok{$}\NormalTok{nbTags <-}\StringTok{ }\OtherTok{NULL}
\NormalTok{data_ACP}\OperatorTok{$}\NormalTok{nbWords <-}\StringTok{ }\OtherTok{NULL}
\NormalTok{data_ACP}\OperatorTok{$}\NormalTok{nbLinks <-}\StringTok{ }\OtherTok{NULL}

\CommentTok{# Ajout du ratio de like et dislike}
\NormalTok{data_ACP}\OperatorTok{$}\NormalTok{ratioLikes <-}\StringTok{ }\NormalTok{data_ACP}\OperatorTok{$}\NormalTok{nbLikes}\OperatorTok{/}\NormalTok{(data_ACP}\OperatorTok{$}\NormalTok{nbLikes }\OperatorTok{+}\StringTok{ }\NormalTok{data_ACP}\OperatorTok{$}\NormalTok{nbDislikes)}
\NormalTok{data_ACP}\OperatorTok{$}\NormalTok{ratioDislikes <-}\StringTok{ }\NormalTok{data_ACP}\OperatorTok{$}\NormalTok{nbDislikes}\OperatorTok{/}\NormalTok{(data_ACP}\OperatorTok{$}\NormalTok{nbLikes }\OperatorTok{+}\StringTok{ }\NormalTok{data_ACP}\OperatorTok{$}\NormalTok{nbDislikes)}

\NormalTok{data_ACP}\OperatorTok{$}\NormalTok{nbLikes <-}\StringTok{ }\OtherTok{NULL}
\NormalTok{data_ACP}\OperatorTok{$}\NormalTok{nbDislikes <-}\StringTok{ }\OtherTok{NULL}

\NormalTok{res.pca =}\StringTok{ }\KeywordTok{PCA}\NormalTok{(data_ACP, }\DataTypeTok{scale.unit=}\OtherTok{TRUE}\NormalTok{, }\DataTypeTok{ncp=}\DecValTok{6}\NormalTok{, }\DataTypeTok{graph=}\NormalTok{T) }
\end{Highlighting}
\end{Shaded}

\begin{verbatim}
## Warning in PCA(data_ACP, scale.unit = TRUE, ncp = 6, graph = T): Missing values
## are imputed by the mean of the variable: you should use the imputePCA function
## of the missMDA package
\end{verbatim}

\includegraphics{projet_files/figure-latex/unnamed-chunk-3-1.pdf}
\includegraphics{projet_files/figure-latex/unnamed-chunk-3-2.pdf}

\hypertarget{acm}{%
\section{ACM}\label{acm}}

\begin{Shaded}
\begin{Highlighting}[]
\NormalTok{data_ACM =}\StringTok{ }\NormalTok{data[}\DecValTok{1}\OperatorTok{:}\DecValTok{50}\NormalTok{,]}
\NormalTok{i=}\DecValTok{0}
\ControlFlowTok{while}\NormalTok{(i}\OperatorTok{<}\KeywordTok{ncol}\NormalTok{(data_ACM))\{}
\NormalTok{  i=i}\OperatorTok{+}\DecValTok{1}
\NormalTok{  data_ACM[,i]=}\KeywordTok{as.factor}\NormalTok{(data_ACM[,i])}
\NormalTok{\}}
\NormalTok{data_ACM}\OperatorTok{$}\NormalTok{index <-}\StringTok{ }\OtherTok{NULL}
\NormalTok{data_ACM}\OperatorTok{$}\NormalTok{nbTags <-}\StringTok{ }\OtherTok{NULL}
\NormalTok{data_ACM}\OperatorTok{$}\NormalTok{nbWords <-}\StringTok{ }\OtherTok{NULL}
\NormalTok{data_ACM}\OperatorTok{$}\NormalTok{nbLinks <-}\StringTok{ }\OtherTok{NULL}
\NormalTok{data_ACM}\OperatorTok{$}\NormalTok{category <-}\StringTok{ }\OtherTok{NULL}
\NormalTok{res.mca =}\StringTok{ }\KeywordTok{MCA}\NormalTok{(data_ACM, }\DataTypeTok{quali.sup =} \KeywordTok{c}\NormalTok{(}\DecValTok{5}\NormalTok{,}\DecValTok{6}\NormalTok{,}\DecValTok{8}\NormalTok{), }\DataTypeTok{graph =} \OtherTok{TRUE}\NormalTok{)}
\end{Highlighting}
\end{Shaded}

\includegraphics{projet_files/figure-latex/unnamed-chunk-4-1.pdf}
\includegraphics{projet_files/figure-latex/unnamed-chunk-4-2.pdf}

\begin{verbatim}
## Warning: Removed 1 rows containing missing values (geom_point).
\end{verbatim}

\begin{verbatim}
## Warning: Removed 1 rows containing missing values (geom_text_repel).
\end{verbatim}

\includegraphics{projet_files/figure-latex/unnamed-chunk-4-3.pdf}

\begin{Shaded}
\begin{Highlighting}[]
\KeywordTok{plot.MCA}\NormalTok{(res.mca, }\DataTypeTok{invisible=}\KeywordTok{c}\NormalTok{(}\StringTok{"var"}\NormalTok{), }\DataTypeTok{cex=}\FloatTok{0.75}\NormalTok{)}
\end{Highlighting}
\end{Shaded}

\includegraphics{projet_files/figure-latex/unnamed-chunk-4-4.pdf}

\hypertarget{afc}{%
\section{AFC}\label{afc}}

\begin{Shaded}
\begin{Highlighting}[]
\NormalTok{data_AFC =}\StringTok{ }\NormalTok{data}
\NormalTok{res.afc =}\StringTok{ }\KeywordTok{CA}\NormalTok{(data_AFC[, }\KeywordTok{c}\NormalTok{(}\DecValTok{2}\OperatorTok{:}\DecValTok{5}\NormalTok{ ,}\DecValTok{12}\OperatorTok{:}\DecValTok{13}\NormalTok{)])}
\end{Highlighting}
\end{Shaded}

\includegraphics{projet_files/figure-latex/unnamed-chunk-5-1.pdf}

\end{document}
